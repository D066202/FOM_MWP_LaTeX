% ---------------------------------------------------------------------------- %
% Einleitung
% ---------------------------------------------------------------------------- %

\section{Einleitung}

Die Einleitung ist wichtig, da sie für den Lesenden den ersten Kontakt mit der Arbeit darstellt und auf das Folgende neugierig machen soll. Die Einführung be- schreibt und begründet die Fragestellung der Arbeit. Sie umfasst eine Hinfüh- rung zum Thema, die Problemstellung, die Zielsetzung der vorliegenden wissenschaftlichen Auseinandersetzung sowie eine Vorgehensbeschreibung. Aus dieser Darstellung werden in der Einleitung Forschungsziele und ggf. Forschungshypothesen hergeleitet und die Relevanz der Forschungsziele begründet. \parencite{brooksbankCognitiveDissonanceRevisited2020}

\newpage