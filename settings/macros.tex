% ---------------------------------------------------------------------------- %
% Makros
% ---------------------------------------------------------------------------- %
%TC:ignore

% Zeilenabstand (vgl Schiefer Leitfaden S. 23)
\newcommand{\zab}{1.5}

% Absatzeinzug
\newcommand{\abs}{0mm} 

% ---------------------------------------------------------------------------- %
% Inhalt Titelseite (vgl Schiefer Leitfaden S. 11)
% ---------------------------------------------------------------------------- %

% Titel und Untertitel
\newcommand{\titel}{Haupttitel}

% Verfasser/Verfasserin
\newcommand{\verfassergen}{Verfasser:}
\newcommand{\verfasser}{Melina Lucia Alt}

% Semester
\newcommand{\semestergen}{Semester:}
\newcommand{\semester}{1}

% Matrikelnummer
\newcommand{\matrgen}{Matrikelnummer: }
\newcommand{\matr}{596869}

% Hochschule
\newcommand{\HS}{FOM Hochschule f\"ur Oekonomie \& Management}
\newcommand{\HSZentrum}{Hochschulzentrum Mannheim}

% Studiengang
\newcommand{\art}{Seminararbeit}
\newcommand{\Studiengang}{im Studiengang Wirtschaftspsychologie}

% Dozent
\newcommand{\dozentgen}{Dozent:}
\newcommand{\dozent}{tbd}

% Modul
\newcommand{\modulgen}{Lehrveranstaltung:}
\newcommand{\modul}{tbd}

% Abgabe
\newcommand{\datumgen}{Datum der Abgabe:}
\newcommand{\datum}{\today}


% ---------------------------------------------------------------------------- %
% Eigene Makros
% ---------------------------------------------------------------------------- %

% Eigene Makros können nach folgendem Schema erstellt werden:
% \newcommand{\yourmacro}{Dein Text} 

% Wörter zählen
\newcommand{\quickwordcount}[1]{
  \immediate\write18{texcount #1.tex > #1-words.tex -inc -sum -0}%
  \input{#1-words} W\"ortern.%
}

%TC:endignore 