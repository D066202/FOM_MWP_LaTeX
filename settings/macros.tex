% ---------------------------------------------------------------------------- %
% Makros
% ---------------------------------------------------------------------------- %

% Zeilenabstand (vgl Schiefer Leitfaden S. 23)
\newcommand{\zab}{1.5}

% Absatzeinzug
\newcommand{\abs}{0mm}

% ---------------------------------------------------------------------------- %
% Inhalt Titelseite
% ---------------------------------------------------------------------------- %

% Nicht verwendete Angaben müssen in setup.tex auskommentiert werden.

% Titel und Untertitel
\newcommand{\haupttitel}{Raus aus der H\"ohle}
\newcommand{\untertitel}{Vom Glanz des Sonnenlichts}

% Verfasser/Verfasserin
\newcommand{\verfassergen}{Verfasser:}
\newcommand{\verfasser}{Melina Lucia Alt}

% Adresse
\newcommand{\strasse}{B1 9}
\newcommand{\ort}{68159 Mannheim}

% Tel
\newcommand{\telgen}{Tel.: }
\newcommand{\tel}{-}

% Mail
\newcommand{\mailgen}{e-Mail: }
\newcommand{\mail}{melina\_lucia.alt@fom-net.com}

% Referent/Referentin
\newcommand{\referentgen}{Referent:}
\newcommand{\referent}{tbd}

% Korreferent/Korreferentin
% \newcommand{\korreferentgen}{Korreferent:}
% \newcommand{\korreferent}{Xenophon}

% Modul
\newcommand{\modulgen}{Modul:}
\newcommand{\modul}{tbd}

% \newcommand{\bearbzeitgen}{Bearbeitungszeitraum:}
% \newcommand{\bearbzeit}{408 bis 403 v. Chr}

% Fusszeile (Ort, Jahr)
% \newcommand{\titlefooter}{Athen, 403 v. Chr.}


% ---------------------------------------------------------------------------- %
% Eigene Makros
% ---------------------------------------------------------------------------- %

% Eigene Makros können nach folgendem Schema erstellt werden:
% \newcommand{\yourmacro}{Dein Text} 

% Wörter zählen
\newcommand{\quickwordcount}[1]{
  \immediate\write18{texcount #1.tex > #1-words.tex -sum -1}%
  \input{#1-words} W\"ortern.%
}