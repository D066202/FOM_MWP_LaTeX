% ---------------------------------------------------------------------------- %
% Packages
% ---------------------------------------------------------------------------- %
%TC:ignore

% Eingabekodierung
\usepackage[utf8]{inputenc}

% Sprache (Deutsch, Neue Rechtschreibung)
\usepackage[ngerman]{babel}

% Arial 11 pt (vgl Schiefer Leitfaden S. 23)
\renewcommand{\familydefault}{\sfdefault}
\usepackage[scaled]{helvet}
\renewcommand\familydefault{\sfdefault} 
\usepackage[T1]{fontenc}
\fontsize{11pt}{2cm}\selectfont

% Ueberschriften (vgl Schiefer Leitfaden S. 23f)
\usepackage{titlesec}
\titleformat{\section}
  {\normalfont\bfseries\filcenter}{\thesection}{1em}{}
\titleformat{\subsection}
  {\normalfont\bfseries}{\thesubsection}{1em}{}
\titleformat{\subsubsection}
  {\normalfont\itshape}{\thesubsubsection}{1em}{}

% Biblatex Settings (APA-Deutsch)
\usepackage[style=apa, autolang=other, url=false]{biblatex}
\DeclareLanguageMapping{ngerman}{ngerman-apa}

% Dient zur Auszeichnung und erstellung von Hyperlinks
\usepackage{url}

\usepackage{ 
	graphicx, 	% Ermöglicht das Einbinden von Bildern
	amsmath,	% Standard Mathe-Paket
	amsfonts,	% Ergänzt mathematische Symbole
	paralist,	% Erweiterung der bereits bestehenden Listenumgebungen
	setspace,	% Zeilenabstand
	geometry,	% Seitenformatierung
	fancyhdr,	% Manipulation von Fuss- und Kopfzeilen
	xcolor,		% Erlaubt das Definieren neuer Farben
	moreverb	% Wörter zählen
}

% Bildunterschriften formatieren
\usepackage[bf]{caption}
\captionsetup{format=plain}

% Layoutanpassungen 
\geometry{
	a4paper,			% Format (überschreibt documentclass)
	footskip=10mm,		% Position Fusszeile (Abstand zum Text)
	headsep=10mm,		% Position Kopfzeile (Abstand zum Text)
	top=30mm,			% vgl Schiefer Leitfaden S. 23
	right=20mm,			% vgl Schiefer Leitfaden S. 23
	bottom=20mm,		% vgl Schiefer Leitfaden S. 23
	left=40mm			% vgl Schiefer Leitfaden S. 23
}

% Mikrotypografie
\usepackage[
	activate={true,nocompatibility},
	final,
	tracking=true,
	kerning=true,
	spacing=true,
	factor=1100,
	stretch=10,
	shrink=10
]{microtype}

\addtokomafont{disposition}{\normalfont}

% Umlaute, Akzente,...
\usepackage[T1]{fontenc}

% Anführungszeichen
\usepackage[autostyle=true, german=swiss]{csquotes}

% Erweiterter Hyperlink Support
\usepackage[hidelinks]{hyperref}


% ---------------------------------------------------------------------------- %
% Optionale Packages und Einstellungen
% ---------------------------------------------------------------------------- %

% Ermöglicht das Einfügen von Notizen mittels \todo{}
\reversemarginpar % Notizen links anzeigen
\setlength{\marginparwidth}{2.2cm}
\usepackage[backgroundcolor=white,linecolor=black,textsize=small]{todonotes}

% Fügt im Literaturverzeichnis "Verfügbar unter" und optional "Letzter Zugriff" 
% zu .bib entries mit url und urldate hinzu.
% \DeclareFieldFormat{formaturl}{Verfügbar unter #1}
% \DeclareFieldFormat{formatdate}{Letzter Zugriff: #1}

%\newbibmacro*{url+urldate}{%
%	\iffieldundef{urlyear}{%
%		\printtext[formaturl]{\printfield{url}}\nopunct%
%	}{%
%		\printtext[formaturl]{\printfield{url}}\adddot\space%
%		\printtext[formatdate]{\printurldate}%
%	}%
%}

\AtEveryBibitem{\clearfield{month}}
\AtEveryCitekey{\clearfield{month}}

% Ergänzt im Literaturverzeichnis den Publisher mit Ort (Ort, Herausgeber)
\DeclareListFormat{publisher}{\printlist{location}\addcomma\space #1}

%TC:endignore