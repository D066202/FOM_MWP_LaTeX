% ============================================================================ %
%
% LaTeX Template
% 
% Autor: Melina Alt 
% GitHub Repository: tbd
% 
% This is free and unencumbered software released into the public domain.
% See https://unlicense.org/
%
% ============================================================================ %


% ---------------------------------------------------------------------------- %
% Dokumenteinstellungen
% ---------------------------------------------------------------------------- %

% Dokumentklasse
\documentclass[a4paper, 11pt, ngerman]{scrartcl}

% Load settings
% ---------------------------------------------------------------------------- %
% Makros
% ---------------------------------------------------------------------------- %
%TC:ignore

% Zeilenabstand (vgl Schiefer Leitfaden S. 23)
\newcommand{\zab}{1.5}

% Absatzeinzug
\newcommand{\abs}{0mm} 

% ---------------------------------------------------------------------------- %
% Inhalt Titelseite (vgl Schiefer Leitfaden S. 11)
% ---------------------------------------------------------------------------- %

% Titel und Untertitel
\newcommand{\titel}{Haupttitel}

% Verfasser/Verfasserin
\newcommand{\verfassergen}{Verfasser:}
\newcommand{\verfasser}{Melina Lucia Alt}

% Semester
\newcommand{\semestergen}{Semester:}
\newcommand{\semester}{1}

% Matrikelnummer
\newcommand{\matrgen}{Matrikelnummer: }
\newcommand{\matr}{596869}

% Hochschule
\newcommand{\HS}{FOM Hochschule f\"ur Oekonomie \& Management}
\newcommand{\HSZentrum}{Hochschulzentrum Mannheim}

% Studiengang
\newcommand{\art}{Seminararbeit}
\newcommand{\Studiengang}{im Studiengang Wirtschaftspsychologie}

% Dozent
\newcommand{\dozentgen}{Dozent:}
\newcommand{\dozent}{tbd}

% Modul
\newcommand{\modulgen}{Lehrveranstaltung:}
\newcommand{\modul}{tbd}

% Abgabe
\newcommand{\datumgen}{Datum der Abgabe:}
\newcommand{\datum}{\today}


% ---------------------------------------------------------------------------- %
% Eigene Makros
% ---------------------------------------------------------------------------- %

% Eigene Makros können nach folgendem Schema erstellt werden:
% \newcommand{\yourmacro}{Dein Text} 

% Wörter zählen
\newcommand{\quickwordcount}[1]{
  \immediate\write18{texcount #1.tex > #1-words.tex -inc -sum -0}%
  \input{#1-words} W\"ortern.%
}

%TC:endignore 
% ---------------------------------------------------------------------------- %
% Packages
% ---------------------------------------------------------------------------- %
%TC:ignore

% Eingabekodierung
\usepackage[utf8]{inputenc}

% Sprache (Deutsch, Neue Rechtschreibung)
\usepackage[ngerman]{babel}

% Arial 11 pt (vgl Schiefer Leitfaden S. 23)
\renewcommand{\familydefault}{\sfdefault}
\usepackage[scaled]{helvet}
\renewcommand\familydefault{\sfdefault} 
\usepackage[T1]{fontenc}
\fontsize{11pt}{2cm}\selectfont

% Ueberschriften (vgl Schiefer Leitfaden S. 23f)
\usepackage{titlesec}
\titleformat{\section}
  {\normalfont\bfseries\filcenter}{\thesection}{1em}{}
\titleformat{\subsection}
  {\normalfont\bfseries}{\thesubsection}{1em}{}
\titleformat{\subsubsection}
  {\normalfont\itshape}{\thesubsubsection}{1em}{}

% Biblatex Settings (APA-Deutsch)
\usepackage[style=apa, autolang=other]{biblatex}
\DeclareLanguageMapping{ngerman}{ngerman-apa}

% Dient zur Auszeichnung und erstellung von Hyperlinks
\usepackage{url}

\usepackage{ 
	graphicx, 	% Ermöglicht das Einbinden von Bildern
	amsmath,	% Standard Mathe-Paket
	amsfonts,	% Ergänzt mathematische Symbole
	paralist,	% Erweiterung der bereits bestehenden Listenumgebungen
	setspace,	% Zeilenabstand
	geometry,	% Seitenformatierung
	fancyhdr,	% Manipulation von Fuss- und Kopfzeilen
	xcolor,		% Erlaubt das Definieren neuer Farben
	moreverb	% Wörter zählen
}

% Bildunterschriften formatieren
\usepackage[bf]{caption}
\captionsetup{format=plain}

% Layoutanpassungen 
\geometry{
	a4paper,			% Format (überschreibt documentclass)
	footskip=10mm,		% Position Fusszeile (Abstand zum Text)
	headsep=10mm,		% Position Kopfzeile (Abstand zum Text)
	top=30mm,			% vgl Schiefer Leitfaden S. 23
	right=20mm,			% vgl Schiefer Leitfaden S. 23
	bottom=20mm,		% vgl Schiefer Leitfaden S. 23
	left=40mm			% vgl Schiefer Leitfaden S. 23
}

% Mikrotypografie
\usepackage[
	activate={true,nocompatibility},
	final,
	tracking=true,
	kerning=true,
	spacing=true,
	factor=1100,
	stretch=10,
	shrink=10
]{microtype}

\addtokomafont{disposition}{\normalfont}

% Umlaute, Akzente,...
\usepackage[T1]{fontenc}

% Anführungszeichen
\usepackage[autostyle=true, german=swiss]{csquotes}

% Erweiterter Hyperlink Support
\usepackage[hidelinks]{hyperref}


% ---------------------------------------------------------------------------- %
% Optionale Packages und Einstellungen
% ---------------------------------------------------------------------------- %

% Ermöglicht das Einfügen von Notizen mittels \todo{}
\reversemarginpar % Notizen links anzeigen
\setlength{\marginparwidth}{2.2cm}
\usepackage[backgroundcolor=white,linecolor=black,textsize=small]{todonotes}
\makeatletter
\renewcommand{\todo}[2][]{%
    \@todo[tickmarkheight=0.2cm,caption={#2}, #1]{\begin{spacing}{0.6}#2\end{spacing}}%
} 
\makeatother 

% Fügt im Literaturverzeichnis "Verfügbar unter" und optional "Letzter Zugriff" 
% zu .bib entries mit url und urldate hinzu.
\DeclareFieldFormat{formaturl}{Verfügbar unter #1}
\DeclareFieldFormat{formatdate}{Letzter Zugriff: #1}

%\newbibmacro*{url+urldate}{%
%	\iffieldundef{urlyear}{%
%		\printtext[formaturl]{\printfield{url}}\nopunct%
%	}{%
%		\printtext[formaturl]{\printfield{url}}\adddot\space%
%		\printtext[formatdate]{\printurldate}%
%	}%
%}

% Ergänzt im Literaturverzeichnis den Publisher mit Ort (Ort, Herausgeber)
\DeclareListFormat{publisher}{\printlist{location}\addcomma\space #1}

%TC:endignore 

% Bibliografie laden
\addbibresource{bibliography.bib}


% ---------------------------------------------------------------------------- %
% Beginn Dokument
% ---------------------------------------------------------------------------- %

\begin{document}

% Setup laden
% ---------------------------------------------------------------------------- %
% Setup
% ---------------------------------------------------------------------------- %

% Dokumentmetadaten
\title{\haupttitel}
\author{\verfasser}
\date{\today}

% Setzt den Zeilenabstand
\setstretch{\zab}

% Setzt den Absatzeinzug
\setlength\parindent{\abs}

% Seitenumbruch für neue Section
\let\stdsection\section
\renewcommand\section{\clearpage\stdsection}

% ---------------------------------------------------------------------------- %
% Kopf- und Fußzeile
% ---------------------------------------------------------------------------- %

% Kopfzeile vgl (vgl Schiefer Leitfaden S. 27)
\fancyhead[L]{}
\fancyhead[C]{\fontsize{10pt}{10pt}\selectfont\thepage}
\fancyhead[R]{}

\fancyfoot[L]{\fontsize{10pt}{10pt}\selectfont}
\fancyfoot[C]{}
\fancyfoot[R]{}


% ---------------------------------------------------------------------------- %
% Formatierung Titelseite
% ---------------------------------------------------------------------------- %

% Nicht verwendete Angaben können hier auskommentiert werden.

\pagestyle{empty}

% Titel und Untertitel
\begin{center}

	\vspace{4cm}
	{\bfseries \haupttitel \par}
	
	\vspace{1cm}
	{\bfseries \untertitel \par}

\end{center}

\vspace*{\fill}

% Angaben zu Verfasser, Referent, ...
\begin{tabbing}

	\verfassergen \hspace{2.5cm} \= \verfasser \\
	\> \strasse \\
	\> \ort \\
	% \> \telgen \tel \\
	\> \mailgen \mail \\
	\\

	\referentgen 
	\> \referent \\
	
%	\korreferentgen
%	\> \korreferent \\
	
	\modulgen
	\> \modul \\
	\\
	
%	\bearbzeitgen
%	\> \bearbzeit
	
	
\end{tabbing}

\ \\
% \centerline{\textbf{\titlefooter}}

% Seitenumbruch
\newpage

\pagestyle{fancy}

\pagenumbering{Roman}



% ---------------------------------------------------------------------------- %
% Inhalt
% ---------------------------------------------------------------------------- %
 
\pagenumbering{Roman}


% Exludiert in TOC (vgl Schiefer Leitfaden S. 13)
\addtocontents{toc}{\protect\setcounter{tocdepth}{0}}

% Neue .tex Dateien können mit \include{} eingebunden werden.
% Nicht verwendete können auskommentiert werden.
% Reihenfolge vgl Schiefer Leitfaden S. 27

%Linksbündig (vgl Schiefer Leitfaden S. 26)
\begin{flushleft}

% Abstract
\setcounter{page}{2}
% ---------------------------------------------------------------------------- %
% Abstract
% ---------------------------------------------------------------------------- %

\section*{Abstract}

\noindent 
Ein Abstract gibt einen kurzen aber möglichst umfassenden Überblick über die Arbeit. Er ist etwa 200 Wörter lang und sollte folgende Informationen beinhalten: Eine Beschreibung der Fragestellung, Kennzeichen der Versuchspersonen, Aufbau und Verlauf der Studie, die wichtigsten Ergebnisse (ggfs. mit statisti- schen Kennwerten) sowie Schlussfolgerungen.

\newpage


% Abbildungsverzeichnis
\phantomsection
\addcontentsline{toc}{section}{Abbildungsverzeichnis} 
\listoffigures
\newpage

% Tabellenverzeichnis
\phantomsection
\addcontentsline{toc}{section}{Tabellenverzeichnis} 
\listoftables
\newpage

% Abkürzungsverzeichnis
\include{content/abkverzeichnis}

\addtocontents{toc}{\protect\setcounter{tocdepth}{3}}

% Inhaltsverzeichnis
\tableofcontents

% Umfang der Arbeit (vgl Schiefer Leitfaden S. 13)
\vfill
Die Thesis hat einen Umfang von \quickwordcount{thesis}\\ 
Grundlage ist der Leitfaden zum wissenschaftlichen Arbeiten in der Wirtschaftspsychologie von Prof. Dr. Gernot Schiefer; Version 2020-09.
 
\newpage


% ---------------------------------------------------------------------------- %
% Einleitung und Hauptteil
% ---------------------------------------------------------------------------- %

% Seitennummerierung (vgl Schiefer Leitfaden S. 27)
\pagenumbering{arabic}

% Erstzeileneinzug (vgl Schiefer Leitfaden S. 26)
\setlength{\parindent}{1.5cm}

% Hauptteil
% ---------------------------------------------------------------------------- %
% Einleitung
% ---------------------------------------------------------------------------- %


\section{Einleitung}

Die Einleitung ist wichtig, da sie für den Lesenden den ersten Kontakt mit der Arbeit darstellt und auf das Folgende neugierig machen soll. Die Einführung beschreibt und begründet die 
Fragestellung der Arbeit. 

Sie umfasst eine Hinführung zum Thema, die Problemstellung, die Zielsetzung der vorliegenden wissenschaftlichen Auseinandersetzung sowie eine Vorgehensbeschreibung. Aus dieser Darstellung werden in der Einleitung Forschungsziele und ggf. Forschungshypothesen hergeleitet und die Relevanz der Forschungsziele begründet. \parencite{cogliserReviewCognitiveDissonance2017}

\newpage
\include{content/theorie}
% ---------------------------------------------------------------------------- %
% Methoden
% ---------------------------------------------------------------------------- %

\section{Methoden}

Beispiel einer Liste:

\begin{itemize} 
	\item irgendwie 
	\item anders
\end{itemize}


\subsection{Methoden Subsection 1}
Nochmals ein wenig Text

\subsubsection{Methoden Subsubsection 1}
Nochmals ein wenig Text‚

\subsubsection{Methoden Subsubsection 2}
Nochmals ein wenig Text

\subsection{Methoden Subsection 2}
Nochmals ein wenig Text
% ---------------------------------------------------------------------------- %
% Ergebnisse
% ---------------------------------------------------------------------------- %

\section{Ergebnisse}

Beispiel einer Liste:

\begin{itemize} 
	\item irgendwie 
	\item anders
\end{itemize}


\subsection{Ergebnisse Subsection 1}
Nochmals ein wenig Text

\subsubsection{Ergebnisse Subsubsection 1}
Nochmals ein wenig Text‚

\subsubsection{Ergebnisse Subsubsection 2}
Nochmals ein wenig Text

\subsection{Ergebnisse Subsection 2}
Nochmals ein wenig Text
% ---------------------------------------------------------------------------- %
% Diskussion
% ---------------------------------------------------------------------------- %

\section{Diskussion}

Beispiel einer Liste:

\begin{itemize} 
	\item irgendwie 
	\item anders
\end{itemize}


\subsection{Diskussion Subsection 1}
Nochmals ein wenig Text

\subsubsection{Diskussion Subsubsection 1}
Nochmals ein wenig Text‚

\subsubsection{Diskussion Subsubsection 2}
Nochmals ein wenig Text

\subsection{Diskussion Subsection 2}
Nochmals ein wenig Text


%Ende des linksbündigen Flattersatzes (beachte TODO)
\end{flushleft}

% ---------------------------------------------------------------------------- %
% Bibliografie
% ---------------------------------------------------------------------------- %

% Standardschrift für URL im Literaturverzeichnis
\urlstyle{same}

% Bibliografie
\phantomsection
\addcontentsline{toc}{section}{Literaturverzeichnis}
\printbibliography[title=Literaturverzeichnis]
\newpage


% ---------------------------------------------------------------------------- %
% Anhang
% ---------------------------------------------------------------------------- %

% Eidesstattliche Erklärung
\include{content/eidesstattlich}

\end{document}
